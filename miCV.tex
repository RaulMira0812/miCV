%%%%%%%%%%%%%%%%%%%%%%%%%%%%%%%%%%%%%%%%%
% Curriculum Vitae
% LaTeX Template
%
% This template has been downloaded from:
% http://www.latextemplates.com
%
% Original author:
% Rensselaer Polytechnic Institute (http://www.rpi.edu/dept/arc/training/latex/resumes/)
%
% Important note:
% This template requires the res.cls file to be in the same directory as the
% .tex file. The res.cls file provides the resume style used for structuring the
% document.
%
%%%%%%%%%%%%%%%%%%%%%%%%%%%%%%%%%%%%%%%%%

%----------------------------------------------------------------------------------------
%	PACKAGES AND OTHER DOCUMENT CONFIGURATIONS
%----------------------------------------------------------------------------------------

\documentclass[margin]{res} % Use the res.cls style

\usepackage{helvet} % Default font is the helvetica postscript font
%\usepackage{newcent} % To change the default font to the new century schoolbook postscript font uncomment this line and comment the one above

\setlength{\textwidth}{5.1in} % Text width of the document

\begin{document}

%----------------------------------------------------------------------------------------
%	NAME AND ADDRESS SECTION
%----------------------------------------------------------------------------------------

\moveleft.5\hoffset\centerline{\large\bf Raul Mira Rodriguez} % Your name at the top
 
\moveleft\hoffset\vbox{\hrule width\resumewidth height 1pt}\smallskip % Horizontal line after name; adjust line thickness by changing the '1pt'
 
\moveleft.5\hoffset\centerline{Duran. Callejon Bolivar 114 y Flor Maria Reinoso} % Your address
\moveleft.5\hoffset\centerline{Duran, Ecuador}
\moveleft.5\hoffset\centerline{(593) 046025070}

%----------------------------------------------------------------------------------------

\begin{resume}

%----------------------------------------------------------------------------------------
%	OBJECTIVE SECTION
%----------------------------------------------------------------------------------------
 
\subsection{OBJECTIVO}  

Ejercer profesionalmente en el campo computacional, con enfasis en el conocimiento de tecnologias de informacion y comunicaciones enfocadas al manejo de audio, video y graficos por computadora, para el desarrollo e innovacion de soluciones multimedios tales como: aplicaciones y contenido interactivos, juegos, simuladores, educacion a distancia, realidad virtual, entre otros.

%----------------------------------------------------------------------------------------
%	EDUCATION SECTION
%----------------------------------------------------------------------------------------

\section{EDUCACION PRIMARIA}

Unidad Educativa Experimental Liceo Aeronautico FAE N.2 "Guayaquil"

\section{EDUCACION SECUNDARIA}

{\sl Bachiller en ciencias, Fisico-Matematico,} 
Unidad Educativa Experimental Liceo Aeronautico FAE N.2 "Guayaquil"

\section{EDUCACION SUPERIOR}

{\sl cursando carrera ,} 5to semestre Ingenieria en ciencias computacionales. Especializacion Sistemas multimedia. \\
Escuela Superior Politecnica del Litoral \\

%----------------------------------------------------------------------------------------
%	COMPUTER SKILLS SECTION
%----------------------------------------------------------------------------------------

\section{COMPUTACION \\ HABILIDADES} 

{\sl Lenguajes \& Programacion:} 
C, C++, Visual Basic. NET, Java \\
{\sl Idiomas :} 
Espanol e Ingles\\
 


\end{resume}
\end{document}